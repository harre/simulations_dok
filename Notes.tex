\chapter{Notes}

\begin{itemize}
\item[29.02.2012]
\texttt{stages\_12} re-rockstarred auf AMD-03 \\
\texttt{stages\_21} rockstarred auf AMD-04 \\
100Mpc $512^3$ jobs: 11410, 15725, 27036, 7755 \\
10 100Mpc ICs generated \\
\textbf{Note: try bigger volumes with NGen-IC} \\ 
added output redshifts derived from \texttt{gadget\_timer.txt} as 
parameter \texttt{outputRedshifts} in .xml file \\ 
Random seeds that do not create cluster like structures at 
32Mpc box: 589, 12170, 13610, 16604, 16749, 17362, 17433, 29666, 32223, 
17595, 22045, 3724, 3183, 4152, 7581, 8502, 10153, 10657, 22946, 14841, 
25060, 29468, 32634
\\
Random seeds that look a little interesting: 15039 $\rightarrow$ rockstarred 
on AMD-03, 26214 $\rightarrow$ rockstarred on AMD-04


\item[28.02.2012]
Successfully started some N-GenIC jobs for comparison of IC generation \\

\item[17.02.2012]
Discussion with Asmus about Stages Cluster $\rightarrow$ try more systematic 
approach to ICs 

\item[15.02.2012]
Galacticus revisision 708 - \texttt{drd5\_r256\_2} not fixed 
$\rightarrow$ E-Mail to Andrew \\
check tomorrow: Galacticus jobs \texttt{fuenfincr256\_1} and 
\texttt{drdx\_3\_r256} \\
\textbf{Note: think about / find a good method for common metadata} \\

\item[14.02.2012]
Wrote E-Mail to Bertschinger. \\

\item[13.02.2012]
Deleted some jobs I started yesterday because they had artificial crosses 
or were practically unconstrained \\
Third simulation \texttt{fuenfincr256\_1} ran through - Galacticus 
restart worked well! \\
\textbf{Note: IC with same seed but higher resolution do not yield the same 
simulation!} $\rightarrow$ started two more test runs from r128 sims to doublecheck \\


\item[12.02.2012]
Updated Galacticus to revision 707 as suggested by Andrew and added parameter 
\texttt{hotHaloOutflowAngularMomentumAlwaysGrows} to xml file. \\
Two of four simulations ran through (copied hdf5 to transfer), 
two crashed $\rightarrow$ try to continue at saved states!

\item[10.02.2012]
wrote E-Mail to Andrew about performance problems and wavelenght computation error 
in \texttt{fuenfincr256\_1} \\
started some runs with higher central delta and broader smoothing lenghts, i.e. 
32/dx and 100/dx; all 128 resolution except second last one (same seed!): 
\begin{verbatim}
83492 0.60500 d31c_1_sta harre   r     02/10/2012 15:19:56 intel.q@astro18  16        
83493 0.60500 d31c_2_sta harre   r     02/10/2012 15:20:37 intel.q@astro29  16        
83494 0.60500 d31c_3_sta harre   r     02/10/2012 15:21:17 intel.q@astro25  16        
83495 0.60500 d51c_sl100 harre   r     02/10/2012 15:23:21 intel.q@astro31  16  
83496 0.54786 d3+3c_sl50 harre   r     02/10/2012 15:37:13 intel.q@astro12  16        
83497 0.60500 d3+3c_sl50 harre   r     02/10/2012 15:39:16 intel.q@astro30  32  
83498 0.60500 d15+3c_sl5 harre   r     02/10/2012 15:44:23 intel.q@astro30  16        
\end{verbatim}

\item[09.02.2012]
\texttt{drd5\_r256} last written to hdf5 file feb 09, 05:07 \\
\texttt{fuenfincr256\_2} last written to hdf5 file feb 06, 03:28 \\
\texttt{drd5\_r256\_2} last written to hdf5 file feb 07, 00:50 \\

\item[02.02.2012]
\texttt{drdx\_h100\_128\_1} run has again severe consistency 
metric problem \\ $\rightarrow$ not clear why \\
upper python script does not work, was commented out again \\
plan: \textbf{move to python scripts in general in order to have
 easier arithmetic calculations} \\   
plan: create new folder structure and remove old simulations $\rightarrow$ done \\

\item[31.01.2012]
note: h=70.3 in galacticus xml input file is expected, consistent tree obviously implies it \\
$\rightarrow$ fixed: changed in markus parameter file for the converter and in xml file \\
$\rightarrow$ question: why not read out? \\
$\rightarrow$ python updateGalacticusStart.py from Markus 

\item[30.01.2012]
new consistenttree with vmax=20

\end{itemize}